%
\section{Conclusions and Related Work}
\label{sec:csr}
We have presented a first attempt of formalizing the operational semantics of the Node.js event-loop asynchronous computation model including some of the more intricate elements (priority callback queues, nested callbacks, closures) of such a programming system.
Following the underlying structured of the event-based loop (inspired to the V8 engine), we have formulated the semantics in terms of an  abstract machine operating on a parametric transition system describing the semantics of the host scripting language. We believe that formal specification and verification of this kind of systems will be more and more important in order to improve the development process of Internet of Things applications, their reliability, and in order to provide non ambiguous documentations of low level details of primitives like those described in this paper. More work has still to be done concerning automation of the verification task e.g. by exploiting approximated algorithms and abstraction for both procedural and functional scripting languages.

There exist several works on formal models of asynchronous programs. 
In \cite{SV06} the authors provide verification algorithms for asynchronous systems modeled as pushdown systems with external memory. The external memory is defined as a multiset of pending procedure calls. Theoretical results on recognizability of Parikh images of context-free language are used to obtain an algorithmic characterization of the reachable set of the resulting model.
The algorithms have been extended to other types of external memory in \cite{CV07}.
Algorithms for liveness properties are studied in \cite{GMR09} and for real-time extensions are given in  \cite{GM09}. A complexity analysis of decidable fragments is given in \cite{GM12}. In \cite{GHR15} the authors consider a general model of event-based systems in which task are maintained in FIFO queues. The focus of their analysis again is providing algorithmic techniques for  different types of restrictions of the model  via reductions to Petri Nets, PDS, and Lossy channel systems. In \cite{EGMR15} the authors define a model for asynchronous programs with task buffers in which events and buffers are dynamically created. Decidable fragments are obtained via reductions to Data nets. 
Differently from the above mentioned work, the goal of the present paper is not that of isolating decidable fragments. We are interested instead in giving a precise semantics to the interplay between  asynchronous architecture like Node.js and scripting languages executed on top of them see e.g. more empirical works like \cite{ASOP16,GMB15}. In this sense we think that, more than restrictions, our framework needs further extensions in order to  capture for instance objects and dynamic memory allocation as done in formal semantics of languages like Javascript \cite{PSR15}. Our validation approach is based on enumeration techniques and partial search similar to tools used for concurrent systems.
