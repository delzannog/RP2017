\section{Formal Reasoning}
\label{sec:fr}
%
In this section we will consider some example of formal reasoning via the transition systems introduced in the previous sections.
For the sake of simplicity, in all example but the last one on closures with state, we will
omit the heap component and consider only environments mapping variables to values.
We will use the complete semantics when considering side effects on closures.

Let us first consider the program defined as follows.
$$
\begin{array}{l}
P=let \ f=(let\ (cb=\lambda x.\ obs(x))\ in\ call(read,cb)\cdot f)\ in\ f()
\end{array}
$$
A possible computation, in which we apply the reduction 
$\tuple{\ell,f}\cdot\tuple{\ell,\epsilon}\equiv \tuple{\ell,f}$, is given below.
$$
\begin{array}{l}
\rho_1=\tuple{\emptyset,\emptyset,\emptyset,\tuple{\epsilon,P},\epsilon,\epsilon,\epsilon,\emptyset}
\rightarrow\\
\rho_2=\tuple{\emptyset,\emptyset,\emptyset,\tuple{\ell,f},\epsilon,\epsilon,\epsilon,\emptyset}
\ s.t.\ \ell\ defines\ f
\rightarrow\\
\rho_3=\tuple{\emptyset,\emptyset,\emptyset,\tuple{\ell,f},\epsilon,\epsilon,\epsilon,\emptyset}
\rightarrow\\
\rho_4=\tuple{\emptyset,\emptyset,\emptyset,\tuple{\ell,let\ cb=\lambda x.obs(x)\ in\ call(read,cb)\cdot f},\epsilon,\epsilon,\epsilon,\emptyset}
\rightarrow\\
\rho_5=\tuple{\emptyset,\emptyset,\emptyset,\tuple{\ell',call(read,cb)\cdot f},
       \epsilon,\epsilon,\epsilon,
			 \emptyset}\ s.t.\ \ell'\ defines\ cb\ \rightarrow\\
\rho_6=\tuple{\emptyset,\emptyset,\emptyset,\tuple{\ell',f},
  \epsilon,\epsilon,\epsilon,
	\{\tuple{\ell',call(read,cb)}\}}
\end{array}
$$
In the configuration $\rho_6$ we can either assume that the asynchronous call is already terminated or
continue with the execution of the current callback.
In the former case we have the following continuation.
$$
\begin{array}{l}
\rho_6 \rightarrow\rho_7=\tuple{\emptyset,\emptyset,\emptyset,\tuple{\ell',f},
              \epsilon,\epsilon,
							\tuple{\ell',cb(d)},\emptyset}\ for\ d\in Val
\end{array}
$$
In the latter case we will add a new frame to the call stack with the body of $f$.
We now observe that, in both cases, the callback stack will never get empty again.
Therefore, the callback $cb$, defined in the local environment $\ell'$, will never be selected for execution even under additional fairness conditions. As a consequence, the transition system will never generate observations during any of its infinitely many possible computations (the termination of the asynchronous call can happen anytime).
This formally explains why in systems like Node.js everytime the main application has a recursive structure  (e.g. to model an infinite loop) in order to make it responsive to external events, it is necessary to  encapsulate the recursive call into invocations of primitives like {\tt setImmediate} and
{\tt nextTick}. To illustrate this idea, let us consider the following modified program.
$$
\begin{array}{l}
P_1=let \ f=(let\ cb=\lambda x.\ obs(x)\ in\ call(read,cb)\cdot setimmediate(f))\ in\ f()
\end{array}
$$
We obtain the following behavior.
$$
\begin{array}{l}
\rho_1=\tuple{\emptyset,\emptyset,\emptyset,\tuple{\epsilon,P_1},\epsilon,\epsilon,\epsilon,\emptyset}
\rightarrow\\
\rho_2=\tuple{\emptyset,\emptyset,\emptyset,\tuple{\ell,f},\epsilon,\epsilon,\epsilon,\emptyset} s.t.\ \ell\ defines\ f
\rightarrow^*\\
\rho_3=
\tuple{\emptyset,\emptyset,\emptyset,\tuple{\ell',setimmediate(f)},\epsilon,\epsilon,\epsilon,\{\tuple{\ell',call(read,cb)}\}}
\ s.t.\ \ell'\ def.\ f,cb
\end{array}
$$
Again we have a bifurcation here. For instance, let us assume that $read$ terminates.
We obtain then the following computation.
$$
\begin{array}{l}
\rho_3
\rightarrow
\rho_4=
\tuple{\emptyset,\emptyset,\emptyset,\tuple{\ell,setimmediate(f)},\epsilon,cb(d),\epsilon,\emptyset}
\rightarrow^*\\
\rho_5=
\tuple{\emptyset,\emptyset,\emptyset,\bot,\epsilon,\tuple{\ell',cb(d)},\tuple{\ell',f},\emptyset}
\rightarrow\\
\rho_6=
\tuple{\emptyset,\emptyset,\emptyset,\tuple{\ell',cb(d)},\epsilon,\tuple{\ell',f},\emptyset}
\rightarrow\\
\rho_7=
\tuple{\emptyset,\emptyset,\emptyset,\tuple{\ell'[x/d],obs(x)},\epsilon,\tuple{\ell',f},\emptyset}
\rightarrow^{d}\\
\rho_8=
\tuple{\emptyset,\emptyset,\emptyset,\tuple{\ell'[x/d],\epsilon},\epsilon,\tuple{\ell',f},\emptyset}
\ldots
\end{array}
$$
It is interesting to observe here that callbacks are evaluated in the environment of the caller. For instance, in this example function $cb$ and $f$ are both defined in $\ell'$, the local environment 
used to initialize the frame associated to the invocation $cb(d)$. In other words, under fairness conditions on the termination of asynchronous callbacks, the execution of program $P_1$ generates infinite traces labeled with an arbitrary sequence of values that correspond to successful termination of read operations.

To understand the difference between \verb+nextTick+ and \verb+setImmediate+, let us first consider the nextTick operation.
$$
\begin{array}{l}
NT=let\ \ f_1=(\lambda d_1.obs(a)),\ f_2=(\lambda d_2.obs(b)),\ f_3=(\lambda z. obs(c))\\
\ \ \ \ \ \ in \ nexttick(f_1)\cdot call(b,f_2)\cdot call(c,f_3) 
\end{array}
$$
We then apply our operational semantics to study its behavior.
$$
\begin{array}{l}
\rho_1=\tuple{\emptyset,\emptyset,\emptyset,\tuple{\epsilon,W},\epsilon,\epsilon,\epsilon,\emptyset}
\rightarrow\\
\rho_2=\tuple{\emptyset,\emptyset,\emptyset,\tuple{\ell,nexttick(f_1)\cdot call(b,f_2)\cdot call(c,f_3)},\epsilon,\epsilon,\epsilon,\emptyset}
\\ s.t.\ \ell\ cont.\ all\ def.
\rightarrow\\
\rho_3=\tuple{\emptyset,\emptyset,\emptyset,\tuple{\ell,call(b,f_2)\cdot call(c,f_3)},\tuple{\ell,f_1},\epsilon,\epsilon,\emptyset}
\rightarrow\\
\rho_4=\tuple{\emptyset,\emptyset,\emptyset,\tuple{\ell,call(b,f_2)},\tuple{\ell,f},\epsilon,\epsilon,\{call(c,f_3)}
\end{array}
$$
We now have possible bifurcations due delays in the termination of $c$.
We could for instance reach one of the following two configurations:
$$
\begin{array}{l}
\rho_5=\tuple{\emptyset,\emptyset,\emptyset,\tuple{\ell,call(c,f_3)},\tuple{\ell,f_1},call(b,f_2),\epsilon,\emptyset}
\\
\rho_5'=\tuple{\emptyset,\emptyset,\emptyset,\bot,\tuple{\ell,f_1},\epsilon,\epsilon,\{call(b,f_2),call(c,f_3)\}}
\end{array}
$$
From the latter we can obtain 
$$
\begin{array}{l}
\rho_6'=\tuple{\emptyset,\emptyset,\emptyset,\bot,\tuple{\ell,f_1}, call(b,f_2)\cdot call(c,f_3),\epsilon,\emptyset}
\\
\rho_6''=\tuple{\emptyset,\emptyset,\emptyset,\bot,\tuple{\ell,f_1},call(b,f_3)\cdot call(c,f_2),\epsilon,\emptyset}
\end{array}
$$
In all cases $f_1$ will be executed before $f_2$ and $f_3$ because of the priority order used to inspect the queue of pending 
calls when the call stack becomes empty, i.e., neither $f_2$ nor $f_3$ can overtake $f_1$.

Now let us consider the same program with $setimmediate$ replacing $nexttick$.
$$
\begin{array}{l}
NT=let\ \ f_1=(\lambda d_1.obs(a)),\ f_2=(\lambda d_2.obs(b)),\ f_3=(\lambda z. obs(c))\\
\ \ \ \ \ \ in \ setimmediate(f_1)\cdot call(b,f_2)\cdot call(c,f_3) 
\end{array}
$$
We then apply our operational semantics to study its behavior.
$$
\begin{array}{l}
\rho_1=\tuple{\emptyset,\emptyset,\emptyset,\tuple{\epsilon,W},\epsilon,\epsilon,\epsilon,\emptyset}
\rightarrow\\
\rho_2=\tuple{\emptyset,\emptyset,\emptyset,\tuple{\ell,setimmediate(f_1)\cdot call(b,f_2)\cdot call(c,f_3)},\epsilon,\epsilon,\epsilon,\emptyset}
\\ s.t.\ \ell\ cont.\ all\ def.
\rightarrow\\
\rho_3=\tuple{\emptyset,\emptyset,\emptyset,\tuple{\ell,call(b,f_2)\cdot call(c,f_3)},\epsilon,\epsilon,
\tuple{\ell,f_1},\emptyset}
\rightarrow\\
\rho_4=\tuple{\emptyset,\emptyset,\emptyset,\tuple{\ell,call(b,f_2)},\epsilon,\epsilon,\tuple{\ell,f_1},
\{call(c,f_3)\}}
\end{array}
$$
We now have possible bifurcations due delays in the termination of $c$.
We could for instance reach one of the following two configurations:
$$
\begin{array}{l}
\rho_5=\tuple{\emptyset,\emptyset,\emptyset,\tuple{\ell,call(c,f_3)},\epsilon,call(b,f_2),\tuple{\ell,f_1},\emptyset}
\\
\rho_5'=\tuple{\emptyset,\emptyset,\emptyset,\bot,\epsilon,\epsilon,\tuple{\ell,f_1},
\{\tuple{\ell,call(b,f_2)},\tuple{\ell,call(c,f_3)}\}}
\end{array}
$$
From the latter we can obtain one of the following configurations:
$$
\begin{array}{l}
\rho_6=\tuple{\emptyset,\emptyset,\emptyset,\tuple{\ell,f_1},\epsilon,
               \epsilon,\{\tuple{\ell,call(b,f_2)}\cdot \tuple{\ell,call(c,f_3)}\}}
\\
\rho_7=\tuple{\emptyset,\emptyset,\emptyset,\bot,\tuple{\ell,call(b,f_2)}\cdot \tuple{\ell,call(c,f_3)},
               \tuple{\ell,f_1},\emptyset}
\\
\rho_8=\tuple{\emptyset,\emptyset,\emptyset,\bot,\tuple{\ell,call(c,f_3)}\cdot \tuple{\ell,call(b,f_2)},
               \tuple{\ell,f_1},\emptyset}
\end{array}
$$

