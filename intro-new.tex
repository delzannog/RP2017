\begin{abstract}
We present the operational semantics of an abstract  machine that models computations of event-based asynchronous programs inspired to the Node.js server-side system, a convenient platform for developing Internet of Things applications. 
The goal of the formal description of Node.js internals is twofold: 
(1) integrating the existing documentation with a more rigorous semantics and 
(2) validating widely used programming and transformation patterns by means of mathematical tools like transition systems. 
Our operational semantics is parametric in the transition system of the host scripting language to mimic the infrastructure of the V8 virtual machine 
where Javascript code is executed on top of the event-based engine provided by the C++ libuv concurrency library.
In this work we focus our attention on priority callback queues, nested callbacks, and closures; these are
widely used Node.js programming features which, however, may render programs difficult to understand, manipulate and validate. 
%For the resulting framework, we propose a property specification language
%based on register automata allowing observation of (infinite) execution traces. 
%The abstract machine and the specification language are used in a  preliminary prototype implementation for bounded verification  based on logic programming.
\end{abstract}
%
\section{Introduction}
Asynchronous programming is getting more and more popular thanks to emergent server-side platforms like Node.js and operating systems for applications that require a constant interaction with the user interface; this programming paradigm reduces the need of controlling concurrency using 
synchronization primitives like lock and monitors. As an example, the execution model of Node.js is based on a single threaded loop used to poll events with different priorities and to serialize the execution of pending tasks by means of callbacks. 
Accordingly, I/O operations are performed through calls to asynchronous functions where callbacks are passed to specify how the computation continues once the corresponding I/O operations completed
asynchronously.

Taking inspiration from previous work \cite{ASOP16,CV07,GM09,GM12,GMB15,GMR09,GHR15,SV06}, in this paper we focus our attention on a formal semantics of asynchronous programs with the following features:  Priority callback queues; Nested callback definitions to model anonymous callbacks; Closures used to propagate the caller scope to the postponed callback invocation.
The formal semantics of the resulting system is given in a parametric form. Namely, we give a presentation that is modular in the transition system  of the host programming language. We use callbacks as a bridge between the two layers as in implementation of scripting languages like Node.js running on virtual machines like Chrome V8.
In our framework a callback is a procedure whose execution is associated with a given event. When the event is triggered by the program or by an external agent, the corresponding callback is added to a queue of pending tasks. In our model the execution of callbacks is controlled by an event-loop. After the complete execution of a callback, the event loop polls the queue and selects the next callback to execute. 

This kind of behavior is typically supported via concurrent executions of I/O bound operations on a pool of worker threads.  To get closer to the Node.js execution model, we model the behavior of the event loop using different phases. More precisely, we provide continuations as a means to define pending tasks with highest priority (they are executed at the end of a callback) as for the {\tt process.nextTick}\footnote{Despite of the name, this is the current semantics of Node.js} operation in Node.js.   Furthermore, we provide postponed callbacks as a means to postpone a callback after the poll phase of standard callbacks (internal, I/O, and networking events). This mechanism is similar to the {\tt setImmediate} operation provided in Node.js. 
As in Node.js nested continuations (a continuation that invokes a continuation) or the enqueuing of tasks  during the poll phase are potential sources of starvation for the event loop. In actual implementations non termination in the poll phase is avoided by imposing a hard limit on the number of callbacks to be executed in each tick. All pending postponed callbacks are executed when the poll phase is quiescent.

{\em Examples}
To illustrate all above mentioned concepts, let us consider some examples taken from the standard \verb+fs+ module supporting file system operations in Node.js.
\begin{verbatim}
var fs = require("fs");
fs.readFile('abc.txt',function(err,data){err||console.log(data);});
console.log("ok");
\end{verbatim}
Function \verb+readFile+ is asynchronous; once the read operation has completed, its continuation is defined by an anonymous callback with two parameters \verb+err+ and \verb+data+ holding, respectively, an optional error object 
and the data read from the file (in case no error occurred, hence, \verb+err+ contains 
\verb+null+).  If the read operation completes successfully, then the callback will print the read data on the standard output, but only after string \verb+ok+.
Let us now consider an example that uses the \verb+events+ module in Node.js for emitting events and registering associated callbacks.
\begin{verbatim}
var EventEmitter = require('events');
var Emitter = new EventEmitter();
var msg = function msg() { console.log('ok'); }
Emitter.on('evt1', msg);
Emitter.emit('evt1');
while (true);
\end{verbatim}
On line 4 the callback \verb+msg+ is registered and associated with events of type \verb+'evt1'+, then  on the subsequent line an event of type \verb+msg+ is emitted; since  
method \verb+emit+ exhibits a synchronous behaviour, the associated callback is immediately executed and the message \verb+ok+ is printed before
the program enters an infinite loop where no other callback will ever be executed. 
We now modify the program above as follows.
\begin{verbatim}
eventEmitter.on('evt1', function () { setImmediate(msg); });
eventEmitter.emit('evt1');
while (true);
\end{verbatim}
Also in this case after the event has been emitted the associated callback is synchronously executed, but then
\verb+setImmediate+ is used to postpone the execution of function \verb+msg+ to the next loop iteration, therefore the program enters immediately the infinite loop without printing \verb+ok+.  System functions like \verb+setImmediate+ are used to interleave the main thread and callbacks. While there exists a rigorous semantics for Javascript, see e.g.\cite{PSR15}, the Node.js on-line documentation \cite{NodeDoc} has not a 
formal counterpart and contains several ambiguities as can be read in \cite{NodeDocELT}, and in several discussions on the meaning of operations like \verb+setImmediate+, 
\verb+nextTick+, etc. see e.g. \cite{d1,d2,d3}.

The combination of asynchronous programming with nested callbacks and closures
may lead to programs with a quite intricate semantics; let us consider, for instance, the following fragment.
\begin{verbatim}
function test(){
   var d = 5;
   var foo = function(){ d = 10; }
   process.nextTick(foo);
   setImmediate(() => { console.log(d) })
}
test(); 
\end{verbatim}
Function \verb+test+ defines a local variable \verb+d+, which, however, is global to  
the inner function \verb+foo+ which is passed as callback to \verb+process.nextTick+;
technically,  \verb+foo+ is called a closure, since it depends on the global variable 
\verb+d+, which is updated by \verb+foo+ itself. Function \verb+process.nextTick+ postpones the execution of \verb+foo+ to the next loop iteration, however \verb+foo+
has higher priority over the callback passed to \verb+setImmediate+ on the following line.
After the call to \verb+test+ is executed, the variable \verb+d+ local to the call is still allocated in the heap since it is referenced by the
closure \verb+foo+; when \verb+foo+ is called, the value of \verb+d+ is updated to \verb+10+,  therefore the execution of \verb+console.log(d)+ 
will print \verb+10+ as result.

In the paper we  propose a formal model for describing computations and to clarify the semantics of this kind of programs.

{\em Plan of the paper}
In Section \ref{sec:am} we present the formal definitions of the operational semantics of a scripting language and of the abstract machine that captures the event-driven behavior of our scripts. Both components are inspired to Node.js. 
Section \ref{sec:fr} illustrates how to apply the operational semantics to reason about asynchronous programs. In Section \ref{sec:psl} we define a specification language for reasoning about computations of the resulting combined framework (abstract machine and semantics of scripting language).
Finally, in Section \ref{sec:csr} we address related work and future research directions.

 








